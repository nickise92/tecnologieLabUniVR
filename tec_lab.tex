\documentclass[11pt]{report}
\usepackage[utf8]{inputenc}
\usepackage[T1]{fontenc}
\usepackage[italian]{babel}
\usepackage{graphicx}
\graphicspath{ {./img/} }
\usepackage{tabularx}
\usepackage{titlesec}
\usepackage{multicol}
\usepackage{tikz}
\usepackage{geometry}
\geometry{
	a4paper, 
	left=20mm,
	right=20mm,
	top=20mm,
	bottom=20mm
}
\usepackage{hyperref}


% Template
\usepackage{noteTemplate}
\usepackage{titlepage}

\begin{document}
\customtitlepage
\newpage
\tableofcontents

\chapter{PostgreSQL}

\section{Introduzione}

PostgreSQL è un RDBMS con alcune funzionalità orientate agli oggetti. E' multipiattaforma, opensource e gestito da un gruppo di volontari. La documentazione si pu\`o trovare sul sito ufficiale di \url{http://www.postgresql.org}.
PostgreSQL implementa gran parte dello standard SQL: la versione 16 implementa quasi tutte le funzioni SQL 2023, per approfondire si rimanda alla \href{https://www.postgresql.org/docs/16/features.html}{documentazione}.

L'interazione tra un utente e le basi di dati gestite da PostgreSQL avviene secondo il modello \emph{client-server}.

\subsection{Server} 
Il daemon postmaster supervisiona tutti i processi specifici di PostgreSQL.
\subsection{Client(s)} 
Qualsiasi programma in grado di gestire l'interazione con postmaster (input di comandi, invio al postmaster, visualizzazione esito comando, ecc).
Il client standard \`e \textbf{psql}, un'applicazione a linea di comando. Esiste un'alternativa interessante: \href{http://www.pgadmin.org}{\textbf{pgAdmin IV}}, un client grafico pi\`u intuitivo.

\subsection{Esempio di connessione al server tramite psql}
\begin{lstlisting}
psql -h db-srv.di.univr.it -U <userid> <nomeid>
Inserisci la password per l`utente <userid>
psql(16.0)
connessione SSL (protocollo: TLSv1.2, cifrario ECDHE-RSA-AES256-GCM-SHA384, 
bit: 256, compressione: disattivato)

Digita 'help' per avere un aiuto
<usrid> =>
\end{lstlisting}

\section{Sintassi SQL}
\subsection{Notazione}
\begin{itemize}
\item \textbf{[} e \textbf{]} indicano parti opzionali, il termine pu\`o non comparire o comparire una sola volta;
\item \textbf{\{} e \textbf{\}} indicano che il termine pu\`o non comparire o comparire un numero arbitrario di volte;
\item L'istruzione \texttt{CREATE TABLE} definisce uno schema di relazione e ne crea un'istanza vuota: attributi, domini e vincoli.

\end{itemize}

\subsection{Creazione di una tabella}

\begin{lstlisting}{language=SQL}
CREATE TABLE tabella {
	attributo dominio [ valoreDefault ] { vincolo }
	{, attributo dominio [valoreDefault } { vincolo } }
	{, vincoloTabella }
]);
\end{lstlisting}















\end{document}